% !TeX root = Phase-1-Instructions.tex
\documentclass[12pt]{article}
\usepackage{graphicx}
\usepackage{float}
\usepackage{tocloft}
\usepackage{listings}
\usepackage{xcolor}
\usepackage{changepage}
\usepackage{hyperref}
\usepackage{tabularx}
\usepackage{datetime}
\usepackage{geometry}

% Set page margins
\geometry{margin=1in}

\newdateformat{sigdate}{\shortmonthname[\THEMONTH]. \THEDAY, \THEYEAR}

\newcommand{\mysection}[1]{\section*{#1}\hrule\vspace{8pt}\addcontentsline{toc}{section}{#1}}

% Define colors for syntax highlighting
\definecolor{codegray}{rgb}{0.5,0.5,0.5}
\definecolor{codepurple}{rgb}{0.58,0,0.82}
\definecolor{backcolour}{rgb}{0.95,0.95,0.92}
\definecolor{codeblue}{rgb}{0.16,0.29,0.68}
\definecolor{codegreen}{rgb}{0.00,0.54,0.25}
\definecolor{codeorange}{rgb}{0.90,0.41,0.05}

% Define C language settings
\lstdefinestyle{cstyle}{
    backgroundcolor=\color{backcolour},
    commentstyle=\color{codegreen},
    keywordstyle=\color{codeblue},
    numberstyle=\tiny\color{codegray},
    stringstyle=\color{codeorange},
    basicstyle=\ttfamily\footnotesize,
    breakatwhitespace=false,
    breaklines=true,
    captionpos=b,
    keepspaces=true,
    numbers=left,
    numbersep=5pt,
    showspaces=false,
    showstringspaces=false,
    showtabs=false,
    tabsize=4,
    language=C,
    emph={int,char,double,float,unsigned,void},
    emphstyle=\color{codeorange},
    morekeywords={alignas,alignof,asm,auto,bool,break,case,catch,char,char16_t,
    char32_t,class,compl,const,const_cast,continue,decltype,default,delete,
    do,double,dynamic_cast,else,enum,explicit,export,extern,false,float,
    for,friend,goto,if,inline,int,long,mutable,namespace,new,noexcept,
    nullptr,operator,private,protected,public,register,reinterpret_cast,
    return,short,signed,sizeof,static,static_assert,static_cast,struct,
    switch,template,this,thread_local,throw,true,try,typedef,typeid,
    typename,union,unsigned,using,virtual,void,volatile,wchar_t,while},
    escapeinside={(*@}{@*)},
    frame=tb
}

\lstdefinestyle{cstyle2}{
    language=C,
    basicstyle=\small\ttfamily,
    keywordstyle=\color{blue},
    commentstyle=\color{green!60!black},
    stringstyle=\color{red},
    numbers=left,
    numberstyle=\tiny,
    numbersep=5pt,
    breaklines=true,
    breakatwhitespace=true,
    tabsize=4,
    frame=single,
    captionpos=b,
    showstringspaces=false,
    escapeinside={(*@}{@*)},
}

% Cover Page
\begin{document}
\begin{titlepage}
    \begin{center}
        \includegraphics[width=0.7\textwidth]{images/SDSU.png}
        \vspace{1cm}

        \large{Department of Computer Science}\\
        {CS 210 Data Structures}\\
        \vspace{3cm}

        \LARGE\textbf{Program 1 Phase 1}\\
        \textbf{Instructions}\\

        \vspace{5cm}

    \end{center}
\end{titlepage}

%Objective
\mysection{Student Objective}
In this assignment, you will develop two implementations of a Priority Queue: one ordered and one unordered.
Both implementations must adhere to specific criteria, ensuring removal operations consistently return the object
of the highest priority that has been in the queue the longest. Additionally, the FIFO order must be maintained
within each priority level.
\\[2mm]
This assignment will progress through three distinct phases. Phase 1 involves the implementation of two Array classes:
\verb|UnorderedArrayList| and \verb|OrderedArrayList|. Following Phase 1, Phase 2 focuses on the development of a Priority Queue
specific to each array class type. Finally, Phase 3 will entail conducting an analysis of your written code, evaluating aspects
such as time complexity, space complexity, and overall clarity.
\\[2mm]
As you tackle this first program, remember it's your chance to grasp the basics of testing, finding edge cases, and mastering debugging.
Embrace the opportunity to learn test-driven development—it's the perfect starting point for your journey in this course!

% %High Level Description
% \section{High Level Description of Project}

\mysection{Phase 1 Instructions}
To begin, read \emph{Separate Compilation}. If you skip this step, you may struggle to understand how the program files work together.
\\[2mm]
Class prototypes defined in \verb|UnorderedArrayList.h| and \verb|OrderedArrayList.h| are provided. Both of these classes are subclasses
of the pure virtual \verb|AbstractList| class. You \textbf{must} review \verb|AbstractList.h| for reference to understand the functionality of
each method. 
\\[2mm]
Phase 1 of the project will consist of the following files. You must use exactly these filenames.
\\

\renewcommand{\arraystretch}{1.5}
\hspace{1cm}\begin{tabular}{@{} ll @{}}
    \verb|AbstractList.h| & The ADT interface (provided). \\
    \verb|OrderedArrayList.h| & The ordered array definition (provided). \\
    \verb|UnorderedArrayList.h| & The unordered array definition (provided). \\
    \verb|OrderedArrayList.cpp| & The ordered array implementation. \\
    \verb|UnorderedArrayList.cpp| & The unordered array implementation. \\
\end{tabular}
\\[4mm]
Your task is to write your own code for the public functions found in \verb|UnorderedArrayList.h| and \verb|OrderedArrayList.h|. You may not
modify \verb|AbstractList.h|.
\\[2mm]
Testing your code is essential and may require more time than writing the classes. Be sure to test combinations of functions.
\newpage\noindent
\mysection{Additional Details}
\begin{itemize}
    \item You may not modify any ADT interface.
    \item You may add private members as needed.
    \item You may add public functions strictly for debugging purposes. These must be removed before submitting.
    \item You may only submit source files specified in the instructions.
    \item All source code file must have your \textbf{name} and \textbf{RedID} at the beginning of the file.
    \item Each method must be as efficient as possible.
    \begin{itemize}
        \item For example: $O(N)$ complexity is unacceptable if the method could be written with $O(\log N)$ complexity.
        \item Hint: For the ordered array, certain methods must use binary search where possible.
    \end{itemize}
    \item By convention, a lower number means a higher priority.
    \item Your implementations may only \verb|#include| the following files/libraries: \verb|"OrderedArrayList.h"|, \verb|"UnorderedArrayList.h"|,
    and \verb|<stdexcept>|. \verb|<iostream>| may be used for debugging purposes but should be removed before submitting.
    \item You are expected to write all the code yourself, you may not use the Standard Library API for any containers.
    \item Your code must not print anything.
    \item Your code should handle all error conditions gracefully. It should never crash.
    \begin{itemize}
        \item Follow the outline in the ADT interface which specify what exceptions to throw.
    \end{itemize}
    \item Your code must compile and run correctly on Gradescope to receive any credit. Testing on edoras may prove to be a helpful exercise.
    \item Minimal tester/driver programs may be provided to help you test your code. You should add more tests and focus heavily on the edge cases.
    \item Allow sufficient time for testing on Gradescope.
\end{itemize}

\mysection{Turning in Phase 1}
To submit phase 1, you must upload the following C++ files to your \verb|Gradescope| account through Canvas. The automated grading platform will run.
You're permitted unlimited submissions, but \emph{don't} use Gradescope as your testing platform. If we see this happening, we will enforce limited submissions.
\\

\renewcommand{\arraystretch}{1.5}
\hspace{1cm}\begin{tabular}{@{} ll}
    \verb|OrderedArrayList.h|\\
    \verb|UnorderedArrayList.h|\\
    \verb|OrderedArrayList.cpp|\\
    \verb|UnorderedArrayList.cpp|\\
\end{tabular}

\mysection{Code Format and Comments}
\begin{itemize}
    \item Proper indentation:
    \begin{itemize}
        \item Consistent use of spaces or tabs for indentation (typically 2 or 4 spaces).
        \item Nested code blocks should be indented accordingly.
    \end{itemize}
    
    \item Consistent use of braces:
    \begin{itemize}
        \item Braces should be used consistently for code blocks, including if statements, loops, and functions.
    \end{itemize}
    
    \item Descriptive variable and function names:
    \begin{itemize}
        \item Variable and function names should be meaningful and descriptive of their purpose.
        \item Use of camelCase for naming.
    \end{itemize}
    
    \item Consistent naming conventions:
    \begin{itemize}
        \item Follow consistent naming conventions for variables, functions, classes, and other identifiers.
    \end{itemize}
    
    \item Proper spacing:
    \begin{itemize}
        \item Use spaces around operators for readability.
        \item Use consistent spacing within expressions and statements.
    \end{itemize}
    
    \item Use of appropriate data types:
    \begin{itemize}
        \item Choose appropriate data types for variables and functions based on their usage and requirements. (We are not concerned about memory usage,
        so do not worry too much about this one.)
    \end{itemize}
    
    \item Avoidance of magic numbers and hardcoded values:
    \begin{itemize}
        \item Use named constants instead of hardcoded values.
        \item Magic numbers should be avoided or properly explained with comments.
    \end{itemize}
    
    \item Proper use of comments:
    \begin{itemize}
        \item Every file should have a header comment with your name and RedID. (Look at example provided in the appendix.)
        \item Every function should have a leading comment. (Look at example provided in the appendix.)
        \item Include comments to explain complex logic, algorithms, or non-trivial code blocks.
        \item Comments should be clear and concise, adding value to the code understanding.
    \end{itemize}
\end{itemize}

\mysection{Cheating Policy}
There is a zero tolerance policy on cheating in this course. You are expected to complete all programming assignments on your own.
Collaboration with other students in the course is not permitted. You may discuss ideas or solutions in general terms with other students,
but you must not exchange code. During the grading process I will examine your code carefully. Anyone caught cheating on a programming
assignment (or on an exam) will receive an "F" in the course, and a referral to Judicial Procedures.

\newpage\noindent
\mysection{Appendix}
\begin{lstlisting}[style=cstyle2, caption=Example Header Comment]
/**
* @file sample.cpp
* @author <Your Name> <Your RedID>
* @brief This is a sample program to show how to comment
* @date YYYY-MM-DD
*/
#include <stdio.h>

int main() {
    printf("Hello, world!");
    return 0;
}
\end{lstlisting}
    
\begin{lstlisting}[style=cstyle2, caption=Example Function Comment]
/**
* Returns the Object at the specified position in this list
* @param index
* @return Object
* @throws invalid_argument if index out of range
*/
int get(int index) {
    function definition...
}
\end{lstlisting}

\end{document}